% Options for packages loaded elsewhere
\PassOptionsToPackage{unicode}{hyperref}
\PassOptionsToPackage{hyphens}{url}
%
\documentclass[
]{article}
\usepackage{amsmath,amssymb}
\usepackage{lmodern}
\usepackage{iftex}
\ifPDFTeX
  \usepackage[T1]{fontenc}
  \usepackage[utf8]{inputenc}
  \usepackage{textcomp} % provide euro and other symbols
\else % if luatex or xetex
  \usepackage{unicode-math}
  \defaultfontfeatures{Scale=MatchLowercase}
  \defaultfontfeatures[\rmfamily]{Ligatures=TeX,Scale=1}
\fi
% Use upquote if available, for straight quotes in verbatim environments
\IfFileExists{upquote.sty}{\usepackage{upquote}}{}
\IfFileExists{microtype.sty}{% use microtype if available
  \usepackage[]{microtype}
  \UseMicrotypeSet[protrusion]{basicmath} % disable protrusion for tt fonts
}{}
\makeatletter
\@ifundefined{KOMAClassName}{% if non-KOMA class
  \IfFileExists{parskip.sty}{%
    \usepackage{parskip}
  }{% else
    \setlength{\parindent}{0pt}
    \setlength{\parskip}{6pt plus 2pt minus 1pt}}
}{% if KOMA class
  \KOMAoptions{parskip=half}}
\makeatother
\usepackage{xcolor}
\IfFileExists{xurl.sty}{\usepackage{xurl}}{} % add URL line breaks if available
\IfFileExists{bookmark.sty}{\usepackage{bookmark}}{\usepackage{hyperref}}
\hypersetup{
  hidelinks,
  pdfcreator={LaTeX via pandoc}}
\urlstyle{same} % disable monospaced font for URLs
\usepackage{longtable,booktabs,array}
\usepackage{calc} % for calculating minipage widths
% Correct order of tables after \paragraph or \subparagraph
\usepackage{etoolbox}
\makeatletter
\patchcmd\longtable{\par}{\if@noskipsec\mbox{}\fi\par}{}{}
\makeatother
% Allow footnotes in longtable head/foot
\IfFileExists{footnotehyper.sty}{\usepackage{footnotehyper}}{\usepackage{footnote}}
\makesavenoteenv{longtable}
\setlength{\emergencystretch}{3em} % prevent overfull lines
\providecommand{\tightlist}{%
  \setlength{\itemsep}{0pt}\setlength{\parskip}{0pt}}
\setcounter{secnumdepth}{-\maxdimen} % remove section numbering
\ifLuaTeX
  \usepackage{selnolig}  % disable illegal ligatures
\fi

\author{}
\date{}

\begin{document}


    
        
        
    
    
    
        
	
    
    高等数学
    讲义 
    
    
    
    
    	 
    		题  目
    		: 
    		 高等数学讲义     
    	 
    		姓  名
    		: 
    		 Elon Li     
    	 
    		日  期
    		: 
    		完成日期     
                  
    

\hypertarget{ux76eeux5f55}{%
\section{目录}\label{ux76eeux5f55}}

\tableofcontents

\hypertarget{ux5e38ux7528ux6570ux5b66ux77e5ux8bc6}{%
\subsection{常用数学知识}\label{ux5e38ux7528ux6570ux5b66ux77e5ux8bc6}}

\hypertarget{ux5e42ux51fdux6570}{%
\subsubsection{幂函数}\label{ux5e42ux51fdux6570}}

\hypertarget{ux7acbux65b9ux76f8ux5173ux516cux5f0f}{%
\paragraph{立方相关公式}\label{ux7acbux65b9ux76f8ux5173ux516cux5f0f}}

立方公式1:

\[(a+b)^3 = a^3+3a^2b+3ab^2+b^3\]

立方公式2:

\[(a-b)^3=a^3-3a^2b+3ab^3-b^3\]

\hypertarget{ux7acbux65b9ux5deeux516cux5f0f}{%
\paragraph{立方差公式}\label{ux7acbux65b9ux5deeux516cux5f0f}}

\[a^3-b^3=(a-b)(a^2+ab+b^2)\]

\hypertarget{ux7acbux65b9ux548cux516cux5f0f}{%
\paragraph{立方和公式}\label{ux7acbux65b9ux548cux516cux5f0f}}

\[a^3+b^3 = (a+b)(a^2-ab+b^2)\]

\hypertarget{ux4e8cux9879ux5f0fux5b9aux7406}{%
\paragraph{二项式定理}\label{ux4e8cux9879ux5f0fux5b9aux7406}}

\[(a+b)^n= \sum_{k=0}^{n}C_n^ka^kb^{n-k}\]

\textbf{注意:} \(C^m_n=\frac{n!}{m!(n-m)!}\)

\hypertarget{nux65b9ux5deeux516cux5f0f--p14ux5fc3ux4e00ux57faux7840ux8bb2ux4e49--}{%
\paragraph{n方差公式}\label{nux65b9ux5deeux516cux5f0f--p14ux5fc3ux4e00ux57faux7840ux8bb2ux4e49--}}

\[a^n-b^n=(a-b)(a^{n-1}+a^{n-2}b+...+ab^{n-2}+b^{n-1})\]

\hypertarget{ux4e09ux89d2ux51fdux6570ux4e0eux53cdux4e09ux89d2ux51fdux6570}{%
\subsubsection{三角函数与反三角函数}\label{ux4e09ux89d2ux51fdux6570ux4e0eux53cdux4e09ux89d2ux51fdux6570}}

\hypertarget{ux4e09ux89d2ux51fdux6570ux91cdux8981ux516cux5f0f}{%
\paragraph{三角函数重要公式}\label{ux4e09ux89d2ux51fdux6570ux91cdux8981ux516cux5f0f}}

\begin{enumerate}
\def\labelenumi{\arabic{enumi}.}
\item
  \(\csc(\alpha)=\frac{1}{\\sin(\alpha)}\)
\item
  \(\sec(\alpha)=\frac{1}{\cos(\alpha)}\)
\item
  \(cot(\alpha)=\frac{1}{\tan(\alpha)}\)
\item
  \(\sin^2(\alpha)+\cos^2(\beta)=1\)
\item
  \(1+\tan^2(\alpha)=\sec^2(\alpha)\)
\item
  \(1+cot^2(\alpha)=csc^2(\alpha)\)
\end{enumerate}

\hypertarget{ux4e24ux89d2ux548cux5deeux516cux5f0f}{%
\paragraph{两角和差公式}\label{ux4e24ux89d2ux548cux5deeux516cux5f0f}}

\begin{enumerate}
\def\labelenumi{\arabic{enumi}.}
\item
  \(\sin(\alpha+\beta)=\sin(\alpha)\cos(\beta)+\cos(\alpha)\sin(\beta)\)
\item
  \(\sin(\alpha-\beta)=\sin(\alpha)\cos(\beta)-\cos(\alpha)\sin(\beta)\)
\item
  \(\cos(\alpha+\beta)=\cos(\alpha)\cos(\beta)-\sin(\alpha)\sin(\beta)\)
\item
  \(\cos(\alpha-\beta)=\cos(\alpha)\cos(\beta)+\sin(\alpha)\sin(\beta)\)
\end{enumerate}

\hypertarget{ux4e8cux500dux89d2ux516cux5f0f}{%
\paragraph{二倍角公式}\label{ux4e8cux500dux89d2ux516cux5f0f}}

\begin{enumerate}
\def\labelenumi{\arabic{enumi}.}
\item
  \(\sin2\alpha=2\sin\alpha \cdot \cos(\alpha)\)
\item
  \(\cos2\alpha=\cos^2\alpha-\sin^2\alpha=2\cos^2\alpha-1=1-2\sin^2\alpha\)
\item
  \(\tan2\alpha=\frac{2\tan\alpha}{1-\tan^2\alpha}\)
\end{enumerate}

\hypertarget{ux964dux6b21ux516cux5f0f}{%
\paragraph{降次公式}\label{ux964dux6b21ux516cux5f0f}}

\begin{enumerate}
\def\labelenumi{\arabic{enumi}.}
\item
  \(\cos^2\alpha = \frac{1+\cos2\alpha}{2}\)
\item
  \(\sin^2\alpha = \frac{1-\cos2\alpha}{2}\)
\end{enumerate}

\hypertarget{ux548cux5deeux5316ux79ef}{%
\paragraph{和差化积}\label{ux548cux5deeux5316ux79ef}}

\begin{enumerate}
\def\labelenumi{\arabic{enumi}.}
\item
  \(\sin\alpha \cdot \cos\beta = \frac{1}{2}[\sin(\alpha+\beta)+sin(\alpha-\beta)]\)
\item
  \(\cos\alpha \cdot \sin\beta = \frac{1}{2}[\sin(\alpha+\beta)-sin(\alpha-\beta)]\)
\item
  \(\cos\alpha \cdot \cos\beta = \frac{1}{2}[\cos(\alpha+\beta)+sin(\alpha-\beta)]\)
\item
  \(\sin\alpha \cdot \sin\beta = \frac{1}{2}[\cos(\alpha+\beta)-sin(\alpha-\beta)]\)
\end{enumerate}

\hypertarget{ux79efux5316ux548cux5dee}{%
\paragraph{积化和差}\label{ux79efux5316ux548cux5dee}}

\begin{enumerate}
\def\labelenumi{\arabic{enumi}.}
\item
  \(\sin{A}+\sin{B} = 2\sin{\frac{A+B}{2}}\cos{\frac{A-B}{2}}\)
\item
  \(\sin{A}-\sin{B} = 2\cos{\frac{A+B}{2}}\sin{\frac{A-B}{2}}\)
\item
  \(\cos{A}+\cos{B} = 2\cos{\frac{A+B}{2}}\cos{\frac{A-B}{2}}\)
\item
  \(\cos{A}-\cos{B} = 2\sin{\frac{A+B}{2}}\sin{\frac{A-B}{2}}\)
\end{enumerate}

\textbf{证明提示:} 令\(A=\alpha+\beta\),\(B = \alpha-\beta\) ,
则\(\alpha = \frac{A+B}{2}\),\(\beta = \frac{A-B}{2}\),然后引入和差化积公式化简得。

\hypertarget{ux8f85ux52a9ux89d2ux516cux5f0f}{%
\paragraph{辅助角公式}\label{ux8f85ux52a9ux89d2ux516cux5f0f}}

\[a\sin x+b\cos x = \sqrt{a^2+b^2}\sin(x+\varphi)\]

\textbf{注意:}

其中\(\sin \varphi = \frac{b}{\sqrt{a^2+b^2}}\),\(\cos \varphi = \frac{a}{\sqrt{a^2+b^2}}\),
\(\tan \varphi = \frac{b}{a}\)

\hypertarget{ux4e07ux80fdux516cux5f0f}{%
\paragraph{万能公式}\label{ux4e07ux80fdux516cux5f0f}}

\[\sin \alpha = \frac{2\tan{\frac{\alpha}{2}}}{1+\tan^2 \frac{\alpha}{2}} = \frac{2t}{1+t^2}\]

\[\cos \alpha = \frac{1-\tan^2 \frac{\alpha}{2}}{1+\tan^2{\frac{\alpha}{2}}}=\frac{1-t^2}{1+t^2}\]

\[\tan \alpha = \frac{2\tan{\frac{\alpha}{2}}}{1-\tan^2\frac{\alpha}{2}}=\frac{2t}{1-t^2}\]

\textbf{说明:} \(t = \tan^2 \frac{\alpha}{2}\)

\hypertarget{ux53cdux4e09ux89d2ux51fdux6570ux5e38ux89c1ux7ed3ux8bba}{%
\paragraph{反三角函数常见结论}\label{ux53cdux4e09ux89d2ux51fdux6570ux5e38ux89c1ux7ed3ux8bba}}

\[\arcsin x + \arccos x = \frac{\pi}{2}\]

\[\arctan x + arccot x = \frac{\pi}{2}\]

\hypertarget{ux5e38ux89c1ux4e0dux7b49ux5f0f}{%
\subsubsection{常见不等式}\label{ux5e38ux89c1ux4e0dux7b49ux5f0f}}

\hypertarget{ux7eddux5bf9ux503cux4e0dux7b49ux5f0f}{%
\paragraph{绝对值不等式}\label{ux7eddux5bf9ux503cux4e0dux7b49ux5f0f}}

\[\vert \vert x \vert + \vert y \vert \vert \le \vert x +y \vert \le \vert x \vert + \vert y \vert\]

\hypertarget{ux5747ux503cux4e0dux7b49ux5f0fux7b97ux6570ux5e73ux5747ux503cuxaux2265uxaux51e0ux4f55ux5e73ux5747ux503c-uxff09}{%
\paragraph{\texorpdfstring{均值不等式(算数平均值\(\ge\)几何平均值
)}{均值不等式(算数平均值\textbackslash ge几何平均值 )}}\label{ux5747ux503cux4e0dux7b49ux5f0fux7b97ux6570ux5e73ux5747ux503cuxaux2265uxaux51e0ux4f55ux5e73ux5747ux503c-uxff09}}

\textbf{注释:}

调和不等式\(\le\)几何平均值 \(\le\)算数平均值\(\le\)平方平均值

\hypertarget{ux6570ux5217}{%
\subsubsection{数列}\label{ux6570ux5217}}

\hypertarget{ux7b49ux6bd4ux6570ux5217}{%
\paragraph{等比数列}\label{ux7b49ux6bd4ux6570ux5217}}

求和公式:

\[S_n = \frac{a_1(1-q_n)}{1-q}\]

\hypertarget{ux7b49ux5deeux6570ux5217}{%
\paragraph{等差数列}\label{ux7b49ux5deeux6570ux5217}}

求和公式:

\[S_n = \frac{n(a_1+a_n)}{2}\]

\hypertarget{ux6570ux5b66ux5f52ux7eb3ux6cd5}{%
\subsubsection{数学归纳法}\label{ux6570ux5b66ux5f52ux7eb3ux6cd5}}

\begin{enumerate}
\def\labelenumi{\arabic{enumi}.}
\item
  验证 \(n = 1 \)成立
\item
  假设 \$n = k成立
\item
  推出\(n = k + 1\)成立
\end{enumerate}

\hypertarget{ux6781ux9650ux4e0eux8fdeux7eed}{%
\subsection{极限与连续}\label{ux6781ux9650ux4e0eux8fdeux7eed}}

\hypertarget{ux6570ux5217ux7684ux6781ux9650}{%
\subsubsection{数列的极限}\label{ux6570ux5217ux7684ux6781ux9650}}

\hypertarget{ux6570ux5217ux6781ux9650ux5b9aux4e49}{%
\paragraph{数列极限定义}\label{ux6570ux5217ux6781ux9650ux5b9aux4e49}}

已知数列\(\{a_n\}\)和常数A,如果对于任意给定的正数\(\varepsilon\)(无论它多么小),都存在正整数\(N\),使得对于\(n> N\)的一切\(a_n\),不等式\(\vert a_n - A \vert < \varepsilon\)
恒成立,则称当
\(n \rightarrow \infty\)时,\{\(a_n\)\}以\(A\)为极限,或\{\(a_n\)\}收敛于A,记为\(\lim\limits_{n \rightarrow \infty} a_n = A\)或\(a_n \rightarrow \infty\).

\hypertarget{ux6781ux9650ux7684ux6027ux8d28-1}{%
\paragraph{极限的性质}\label{ux6781ux9650ux7684ux6027ux8d28-1}}

\begin{enumerate}
\def\labelenumi{\arabic{enumi}.}
\item
  唯一性
\item
  全局有界性
\item
  保号性

  \begin{itemize}
  \item
    如果\(\lim\limits_{n \rarr \infty}a_n = A\)且\(A > 0(A< 0)\),那么存在正整数\(N > 0\),当\(n > N\)时,都有\(a_n > \frac{A}{2} >0(a_n < \frac{A}{2}<0)\).
  \end{itemize}
\item
  数列收敛,其任意子数列也收敛
\end{enumerate}

\hypertarget{ux51fdux6570ux7684ux6781ux9650}{%
\subsubsection{函数的极限}\label{ux51fdux6570ux7684ux6781ux9650}}

\hypertarget{ux51fdux6570ux6781ux9650ux7684ux5b9aux4e49}{%
\paragraph{函数极限的定义}\label{ux51fdux6570ux6781ux9650ux7684ux5b9aux4e49}}

\hypertarget{xux2192ux221euxaux65f6uxafxuxaux7684ux6781ux9650ux7684ux5b9aux4e49}{%
\subparagraph{\texorpdfstring{\(x \rarr \infty\)时\(f(x)\)的极限的定义}{x \textbackslash rarr \textbackslash infty时f(x)的极限的定义}}\label{xux2192ux221euxaux65f6uxafxuxaux7684ux6781ux9650ux7684ux5b9aux4e49}}

设函数\(f(x)\)在\({\vert x\vert}\)
充分大时有定义,\(A\)为一常数,如果对于任意给定的\(\varepsilon > 0\),都存在一个正数\(N\),使得适合不等式\({\vert x\vert} > N\)的一切\(x\)所对应的函数值\(f(x)\)都满足

\[{\vert f(x) - A\vert} < \varepsilon\]

则称当\(x \ \rarr \infty\)时,\(f(x)\)以\(A\)为极限,记为
\(\lim\limits_{x \rarr \infty} f(x) =A\)或\(f(x) \rarr A(x \rarr \infty)\).

\hypertarget{xux2192x0uxaux65f6uxafxuxaux7684ux6781ux9650ux7684ux5b9aux4e49}{%
\subparagraph{\texorpdfstring{\(x \rarr x_0\)时\(f(x)\)的极限的定义}{x \textbackslash rarr x\_0时f(x)的极限的定义}}\label{xux2192x0uxaux65f6uxafxuxaux7684ux6781ux9650ux7684ux5b9aux4e49}}

设有函数\(f(x)\)在\(x_0\)点的,某一去心邻域内有定义,\(A\)为一常数,如果对于任意给定的\(\varepsilon > 0\),都存在一个正数\(\delta>0\),使得适合不等式\({\vert x-x_0\vert} > \delta\)的一切\(x\)所对应的函数值\(f(x)\)都满足

\[{\vert f(x) - A\vert} < \varepsilon\]

则称当\(x \ \rarr x_0\)时,\(f(x)\)以\(A\)为极限,记为
\(\lim\limits_{x \rarr x_0} f(x) =A\)或\(f(x) \rarr A(x \rarr x_0)\).

\hypertarget{ux6781ux9650ux7684ux6027ux8d28-2}{%
\paragraph{极限的性质}\label{ux6781ux9650ux7684ux6027ux8d28-2}}

\begin{enumerate}
\def\labelenumi{\arabic{enumi}.}
\item
  唯一性
\item
  有界性
\item
  局部保号性

  \begin{itemize}
  \item
    设\(\lim\limits_{n \rarr \infty}f(x) = A\),且\(A>0\)(或\(A<0\)),则必存在\(\mathring{U}(x_0)\),使得\(\forall x \in \mathring{U}(x_0)\),都有\(f(x)>\frac{A}{2}>0\)(或\(f(x)<\frac{A}{2}<0\).
  \item
    设\(\lim\limits_{x \rarr x_0} f(x) =A\)且在\(\mathring{U}(x_0)\)内\(f(x) \ge 0 \)或\(f(x) \le 0\),则\(A \ge 0\)或(\(A \le 0 \)).
  \end{itemize}
\end{enumerate}

\hypertarget{ux6781ux9650ux5b58ux5728ux51c6ux5219ux548cux4e24ux4e2aux91cdux8981ux6781ux9650}{%
\subsubsection{极限存在准则和两个重要极限}\label{ux6781ux9650ux5b58ux5728ux51c6ux5219ux548cux4e24ux4e2aux91cdux8981ux6781ux9650}}

\hypertarget{ux5939ux903cux51c6ux5219}{%
\paragraph{夹逼准则}\label{ux5939ux903cux51c6ux5219}}

\hypertarget{ux6570ux5217ux5939ux903cux51c6ux5219}{%
\subparagraph{数列夹逼准则}\label{ux6570ux5217ux5939ux903cux51c6ux5219}}

如果数列\(\{a_n\},\{b_n\}\)及\(\{c_n\}\)满足以下条件:

\begin{enumerate}
\def\labelenumi{\arabic{enumi}.}
\item
  从某一项起,即\(\exists\)正整数\(N\),当\(n > N\)时,有\(b_n \le a_n \le c_n\);
\item
  \(\lim\limits_{n \rarr \infty} b_n = a, \lim\limits_{n \rarr \infty}c_n=a\);
\end{enumerate}

那么数列\(\{a_n\}\)的极限存在,且\(\lim\limits_{n \rarr \infty}a_n = a\).

\hypertarget{ux51fdux6570ux5939ux903cux51c6ux5219}{%
\subparagraph{函数夹逼准则}\label{ux51fdux6570ux5939ux903cux51c6ux5219}}

如果函数\(f(x),g(x)\)及\(h(x)\)满足以下条件:

\begin{enumerate}
\def\labelenumi{\arabic{enumi}.}
\item
  当\(x \in \mathring{U}(x_0,r)\)或\(\vert x \vert > M\)时,\(g(x)\le f(x)\le h(x)\);
\item
  \(\lim\limits_{n \rarr \infty} g(x) = A, \lim\limits_{n \rarr \infty}h(x)=A\);
\end{enumerate}

那么函数\(f(x)\)的极限存在,且\(\lim\limits_{n \rarr \infty}f(x) = A\).

\hypertarget{ux5355ux8c03ux6709ux754cux51c6ux5219}{%
\paragraph{单调有界准则}\label{ux5355ux8c03ux6709ux754cux51c6ux5219}}

\hypertarget{ux6570ux5217ux5355ux8c03ux6709ux754cux51c6ux5219}{%
\subparagraph{数列单调有界准则}\label{ux6570ux5217ux5355ux8c03ux6709ux754cux51c6ux5219}}

如果数列\(\{a_n\}\)满足\(a_1 \le a_2 \le \cdots \le a_n \le \cdots\),称为单调增数列;满足\(a_1 \ge a_2 \ge \cdots \ge a_n \ge \cdots\)
,称为单调减函数。

极限存在的单调有界准则:\textbf{若单调数列\(\{a_n\}\)是有界的,则\(\lim\limits_{n \rarr \infty}a_n \)存在}。

\hypertarget{ux51fdux6570ux5355ux8c03ux6709ux754cux51c6ux5219}{%
\subparagraph{函数单调有界准则}\label{ux51fdux6570ux5355ux8c03ux6709ux754cux51c6ux5219}}

设函数\(f(x)\)在点\(x_{0}\)的某个左邻域内单调并且有界,则\(f(x)\)在\(x_0 \)的左极限\(f(x_0^-)\)必定存在。

\hypertarget{ux91cdux8981ux6781ux9650}{%
\paragraph{重要极限}\label{ux91cdux8981ux6781ux9650}}

\[\lim\limits_{n \rarr \infty} \frac{\sin x}{x} = 1\]

\[\lim\limits_{n \rarr \infty} (1+\frac{1}{x})^x = e\]

\hypertarget{ux65e0ux7a77ux5c0fux7684ux6bd4ux8f83}{%
\subsubsection{无穷小的比较}\label{ux65e0ux7a77ux5c0fux7684ux6bd4ux8f83}}

\hypertarget{ux65e0ux7a77ux5c0fux56e0ux5b50}{%
\paragraph{无穷小因子}\label{ux65e0ux7a77ux5c0fux56e0ux5b50}}

\hypertarget{ux9ad8ux9636ux65e0ux7a77ux5c0f}{%
\subparagraph{高阶无穷小}\label{ux9ad8ux9636ux65e0ux7a77ux5c0f}}

如果\(\lim\limits \frac{\beta}{\alpha} = 0\),就说\(\beta\)是比\(\alpha\)高阶的无穷小,记为\(\beta = o(\alpha)\)。

\hypertarget{ux4f4eux9636ux65e0ux7a77ux5c0f}{%
\subparagraph{低阶无穷小}\label{ux4f4eux9636ux65e0ux7a77ux5c0f}}

如果\(\lim\limits = \frac{\beta}{\alpha} = \infty\),就说\(\beta\)是比\(\alpha\)低阶的无穷小。

\hypertarget{ux540cux9636ux65e0ux7a77ux5c0f}{%
\subparagraph{同阶无穷小}\label{ux540cux9636ux65e0ux7a77ux5c0f}}

如果\(\lim\limits = \frac{\beta}{\alpha} = C\),就说\(\beta\)是比\(\alpha\)同阶的无穷小,特别的,当\(C=1\)时,就说\(\beta \)与\(\alpha \)是等价无穷小,记为
\(\alpha \sim \beta\)。

\hypertarget{kuxaux9636ux65e0ux7a77ux5c0f}{%
\subparagraph{\texorpdfstring{\(k\)阶无穷小}{k阶无穷小}}\label{kuxaux9636ux65e0ux7a77ux5c0f}}

如果\(\lim\limits = \frac{\beta}{\alpha^k} = C\),就说\(\beta\)是比\(\alpha\)的\(k\)阶的无穷小。

\hypertarget{ux7b49ux4ef7ux65e0ux7a77ux5c0f}{%
\subparagraph{等价无穷小}\label{ux7b49ux4ef7ux65e0ux7a77ux5c0f}}

\begin{enumerate}
\def\labelenumi{\arabic{enumi}.}
\item
  \(x \rarr 0\)时:

  \begin{itemize}
  \item
    \(\sin x \sim x,\tan x \sim x , \arctan x \sim x ,1 - \cos x \sim \frac{1}{2} x^2, \ln(1+x) \sim x\)
  \item
    \(e^x - 1 \sim x, (1+x)^\alpha \sim \alpha x\)
  \end{itemize}
\item
  \(x \rarr 1\)时: \(\ln(x) \sim x-1\)
\end{enumerate}

\hypertarget{ux7b49ux4ef7ux65e0ux7a77ux5c0fux539fux5219}{%
\paragraph{等价无穷小原则}\label{ux7b49ux4ef7ux65e0ux7a77ux5c0fux539fux5219}}

\begin{enumerate}
\def\labelenumi{\arabic{enumi}.}
\item
  整体乘积
\item
  得是无穷小
\end{enumerate}

\hypertarget{ux975eux96f6ux56e0ux5b50ux5e26ux5165ux539fux5219}{%
\paragraph{非零因子带入原则}\label{ux975eux96f6ux56e0ux5b50ux5e26ux5165ux539fux5219}}

\begin{enumerate}
\def\labelenumi{\arabic{enumi}.}
\item
  整体乘积
\item
  该因子极限不能为0
\end{enumerate}

\hypertarget{ux9ad8ux9636ux65e0ux7a77ux5c0fux8fd0ux7b97}{%
\paragraph{高阶无穷小运算}\label{ux9ad8ux9636ux65e0ux7a77ux5c0fux8fd0ux7b97}}

\begin{enumerate}
\def\labelenumi{\arabic{enumi}.}
\item
  \(\lim \limits _{x \rarr 0} \frac{o(x)}{x} = 0\)
\item
  \(o(x^m) \pm o(x^n) = o(x^p)\),其中\(o=min\{m,n\}\)(高阶吸收低阶)
\item
  \(o(x^m) \cdot o(x^n) = o(x^{m+n})\)
\item
  \(x^m \cdot o(x^n) = o(x^{m+n})\)
\item
  \([o(x^m)]^n=o(x^{mn})\)
\item
  \(\frac{o(x^m)}{x^n}= o(x^{m-n})\),要求\(m\ge n\)
\end{enumerate}

\hypertarget{ux51fdux6570ux7684ux8fdeux7eedux6027ux4e0eux95f4ux65adux70b9}{%
\subsubsection{函数的连续性与间断点}\label{ux51fdux6570ux7684ux8fdeux7eedux6027ux4e0eux95f4ux65adux70b9}}

\hypertarget{ux8fdeux7eedux6027ux7684ux5b9aux4e49}{%
\paragraph{连续性的定义}\label{ux8fdeux7eedux6027ux7684ux5b9aux4e49}}

\hypertarget{ux589eux91cfux7684ux5b9aux4e49}{%
\subparagraph{增量的定义}\label{ux589eux91cfux7684ux5b9aux4e49}}

函数\(y = f(x)\)的增量(或者改变量)\(\Delta y\):
设函数\(y = f(x)\)在\(x_0\)的某领域\(U(x_0)\)内有定义,自变量\(x\)从\(x_0 \)变化到\(x_0 + \Delta x  \in U(x_0)\),函数\(y=f(x)\)从相应的\(f(x_0)\)变化到\(f(x_0 + \Delta x)\),因此\(y = f(x)\)在\(x_0\)
点处的增量为\(\Delta y =f(x_0 + \Delta x)- f(x_0)\).

\hypertarget{ux51fdux6570ux8fdeux7eedux7684ux5b9aux4e49}{%
\paragraph{函数连续的定义}\label{ux51fdux6570ux8fdeux7eedux7684ux5b9aux4e49}}

设\(y = f(x)\)在\(U(x_0)\)内有定义,\(x_0+\Delta x \in U(x_0)\),如果\(\Delta x = x - x_0 \rarr 0\)时,对应的函数的增量\(\Delta y = f(x_0+ \Delta x)-f(x_0) \rarr 0\),则称
\(y = f(x)\)在\(x_0\)点连续,即\(\lim \limits_{\Delta x \rarr \infty }\Delta y = 0\),
称\(y = f(x)\)在\(x_0\)点连续。

\hypertarget{ux95f4ux65adux70b9}{%
\paragraph{间断点}\label{ux95f4ux65adux70b9}}

\hypertarget{ux95f4ux65adux70b9ux7684ux5b9aux4e49}{%
\subparagraph{间断点的定义}\label{ux95f4ux65adux70b9ux7684ux5b9aux4e49}}

若函数\(y = f(x)\)在\(x_0\)点不连续,则称\(x_0\)为\(y =f(x)\)的间断点。

函数\(f(x)\)在\(x_0\)点连续\(\lim \limits_{x \rarr x_0} f(x) \Leftrightarrow f(x_0) \),要求

\begin{enumerate}
\def\labelenumi{\arabic{enumi}.}
\item
  \(f(x)\)在\(x_0 \)有定义:
\item
  \(\lim \limits_{x \rarr x_0}f(x)\)存在,即\(\lim\limits_{x \rarr x_0^-}f(x),\lim\limits_{x \rarr x_0^+}f(x)\)存在且相等
\item
  \(\lim\limits_{x \rarr x_0^-}f(x)=\lim\limits_{x \rarr x_0^+}f(x)=f(x_0)\)
\end{enumerate}

若以上的三个条件有一个不满足,则\(x_0\)为函数\(f(x)\)的间断点。

\hypertarget{ux95f4ux65adux70b9ux7684ux79cdux7c7b}{%
\subparagraph{间断点的种类}\label{ux95f4ux65adux70b9ux7684ux79cdux7c7b}}

第一类间断点

\begin{enumerate}
\def\labelenumi{\arabic{enumi}.}
\item
  跳跃间断点
\item
  可去间断点
\end{enumerate}

第二类间断点

\begin{enumerate}
\def\labelenumi{\arabic{enumi}.}
\item
  无穷间断点
\item
  震荡间断点
\end{enumerate}

\hypertarget{ux51fdux6570ux8fdeux7eedux5b9aux7406}{%
\paragraph{函数连续定理}\label{ux51fdux6570ux8fdeux7eedux5b9aux7406}}

\hypertarget{ux6700ux503cux4e0eux6709ux754cux6027ux5b9aux7406}{%
\subparagraph{最值与有界性定理}\label{ux6700ux503cux4e0eux6709ux754cux6027ux5b9aux7406}}

设函数\(f(x)\)在闭区间\([a,b]\)上连续,则\(f(x)\)在该区间必有最大值和最小值且有界。

\hypertarget{ux96f6ux70b9ux5b9aux7406}{%
\subparagraph{零点定理}\label{ux96f6ux70b9ux5b9aux7406}}

设\(f(x)\)在区间\([a,b]\)上连续,若\(f(a)\)与\(f(b)\)异号(即\(f(a)f(b)<0\)),则连续曲线\(y=f(x)\)与\(x\)轴至少有一个交点,即方程\(f(x) = 0\)在区间\((a,b)\)至少有一个实根\(\zeta\)(也就是函数\(f(x)\)至少有一个零点)。

\hypertarget{ux4ecbux8d28ux5b9aux7406}{%
\subparagraph{介质定理}\label{ux4ecbux8d28ux5b9aux7406}}

设函数\(f(x)\)在闭区间\([a,b]\)上连续,则对任一满足\(f(a)<c<f(b)\)(或\(f(b)<c<f(a)\))的\(c\),在开区间\((a,b)\)内至少存在一点\(\zeta\),使得\(f(\zeta) = c(a<\zeta<b)\)。

\textbf{介质定理推论}

设函数\(f(x)\)在闭区间\([a,b]\)上连续,则对任一满足\(m<c<M\)(\(m\)和\(M\)分别是函数\(f(x)\)在\([a,b]\)上的最小值和最大值)的\(c\),在开区间\([a,b]\)内至少存在一点\(\zeta\),使得\(f(\zeta) = c(a\le\zeta\le b)\)。

\hypertarget{ux5bfcux6570ux548cux5faeux5206}{%
\subsection{导数和微分}\label{ux5bfcux6570ux548cux5faeux5206}}

\hypertarget{ux5bfcux6570ux7684ux6982ux5ff5}{%
\subsubsection{导数的概念}\label{ux5bfcux6570ux7684ux6982ux5ff5}}

设函数\(y = f(x)\)在点\(x_0\)的某邻域内有定义,当自变量\(x\)在\(x_0\)处有增量\(\Delta x\)(\(x_0+\Delta x\)仍然在该领域内),函数相应地有增量\(\Delta y=f(x_0+\Delta x)- f(x_0)\).

如果极限\(\lim\limits_{\Delta x \rarr 0} \frac{\Delta y}{\Delta x} = \lim\limits_{\Delta x \rarr 0} \frac{f(x_0+\Delta x)-f(x_0)}{\Delta x}\)存在,则称函数\(y = f(x)\)在点\(x_0\)处可导,此极限值为函数\(y=f(x)\)在点\(x_0\)处的导数,记为:\(f^{'}(x_0)、y^{'}\vert_{x=x0}、\frac{dy}{dx} \vert_{x=x_0}、\frac{df(x)}{dx}\vert_{x=x_0}\)。

\hypertarget{ux5bfcux6570ux7684ux5145ux8981ux6761ux4ef6}{%
\paragraph{导数的充要条件}\label{ux5bfcux6570ux7684ux5145ux8981ux6761ux4ef6}}

\hypertarget{ux53efux5bfcux7684ux5145ux8981ux6761ux4ef6}{%
\paragraph{可导的充要条件}\label{ux53efux5bfcux7684ux5145ux8981ux6761ux4ef6}}

\hypertarget{ux51fdux6570ux7684ux6c42ux5bfcux6cd5ux5219}{%
\subsubsection{函数的求导法则}\label{ux51fdux6570ux7684ux6c42ux5bfcux6cd5ux5219}}

\hypertarget{ux5bfcux6570ux6c42ux5bfcux6cd5ux5219}{%
\paragraph{导数求导法则}\label{ux5bfcux6570ux6c42ux5bfcux6cd5ux5219}}

\hypertarget{ux53cdux51fdux6570ux7684ux6c42ux5bfcux6cd5ux5219}{%
\paragraph{反函数的求导法则}\label{ux53cdux51fdux6570ux7684ux6c42ux5bfcux6cd5ux5219}}

\hypertarget{ux57faux672cux6c42ux5bfcux516cux5f0f}{%
\paragraph{基本求导公式}\label{ux57faux672cux6c42ux5bfcux516cux5f0f}}

\hypertarget{ux9690ux51fdux6570ux6c42ux5bfcux6cd5ux5219}{%
\paragraph{隐函数求导法则}\label{ux9690ux51fdux6570ux6c42ux5bfcux6cd5ux5219}}

\hypertarget{ux5faeux5206ux4e0eux5bfcux6570ux7684ux5173ux7cfb}{%
\subsubsection{微分与导数的关系}\label{ux5faeux5206ux4e0eux5bfcux6570ux7684ux5173ux7cfb}}

\hypertarget{ux57faux672cux521dux7b49ux51fdux6570ux7684ux5faeux5206ux516cux5f0f}{%
\paragraph{基本初等函数的微分公式}\label{ux57faux672cux521dux7b49ux51fdux6570ux7684ux5faeux5206ux516cux5f0f}}

\hypertarget{ux548cux5deeux79efux5546ux7684ux5faeux5206ux8fd0ux7b97ux6cd5ux5219}{%
\paragraph{和、差、积、商的微分运算法则}\label{ux548cux5deeux79efux5546ux7684ux5faeux5206ux8fd0ux7b97ux6cd5ux5219}}

\hypertarget{ux590dux5408ux51fdux6570ux5faeux5206ux6cd5ux5219}{%
\paragraph{复合函数微分法则}\label{ux590dux5408ux51fdux6570ux5faeux5206ux6cd5ux5219}}

\hypertarget{ux5faeux5206ux7684ux51e0ux4f55ux610fux4e49}{%
\paragraph{微分的几何意义}\label{ux5faeux5206ux7684ux51e0ux4f55ux610fux4e49}}

\hypertarget{ux6781ux503cux95eeux9898}{%
\subsubsection{极值问题}\label{ux6781ux503cux95eeux9898}}

\hypertarget{ux51fdux6570ux5355ux8c03ux6027ux5224ux522b}{%
\paragraph{函数单调性判别}\label{ux51fdux6570ux5355ux8c03ux6027ux5224ux522b}}

\hypertarget{ux51fdux6570ux6781ux503cux6c42ux6cd5}{%
\paragraph{函数极值求法}\label{ux51fdux6570ux6781ux503cux6c42ux6cd5}}

\hypertarget{ux6781ux503cux7684ux5145ux5206ux6027}{%
\paragraph{极值的充分性}\label{ux6781ux503cux7684ux5145ux5206ux6027}}

\hypertarget{ux62d0ux70b9ux95eeux9898}{%
\paragraph{拐点问题}\label{ux62d0ux70b9ux95eeux9898}}

\hypertarget{ux6e10ux8fdbux7ebfux95eeux9898}{%
\paragraph{渐进线问题}\label{ux6e10ux8fdbux7ebfux95eeux9898}}

\hypertarget{ux5faeux5206ux4e2dux503cux5b9aux7406ux548cux5bfcux6570ux5e94ux7528}{%
\subsection{微分中值定理和导数应用}\label{ux5faeux5206ux4e2dux503cux5b9aux7406ux548cux5bfcux6570ux5e94ux7528}}

\hypertarget{ux5faeux5206ux4e2dux503cux5b9aux7406}{%
\subsubsection{微分中值定理}\label{ux5faeux5206ux4e2dux503cux5b9aux7406}}

\hypertarget{ux8d39ux9a6cux5b9aux7406}{%
\paragraph{费马定理}\label{ux8d39ux9a6cux5b9aux7406}}

\hypertarget{ux7f57ux5c14ux5b9aux7406}{%
\paragraph{罗尔定理}\label{ux7f57ux5c14ux5b9aux7406}}

\hypertarget{ux62c9ux683cux6717ux65e5ux4e2dux503cux5b9aux7406}{%
\paragraph{拉格朗日中值定理}\label{ux62c9ux683cux6717ux65e5ux4e2dux503cux5b9aux7406}}

\hypertarget{ux67efux897fux4e2dux503cux5b9aux7406}{%
\paragraph{柯西中值定理}\label{ux67efux897fux4e2dux503cux5b9aux7406}}

\hypertarget{ux6d1bux5fc5ux8fbeux6cd5ux5219}{%
\subsubsection{洛必达法则}\label{ux6d1bux5fc5ux8fbeux6cd5ux5219}}

\hypertarget{00uxaux578bux6d1bux5fc5ux8fbeux6cd5ux5219}{%
\paragraph{\texorpdfstring{\(\frac{0}{0}\)型洛必达法则}{\textbackslash frac\{0\}\{0\}型洛必达法则}}\label{00uxaux578bux6d1bux5fc5ux8fbeux6cd5ux5219}}

\hypertarget{ux221eux221euxaux578bux6d1bux5fc5ux8fbeux6cd5ux5219}{%
\paragraph{\texorpdfstring{\(\frac{\infty}{\infty}\)型洛必达法则}{\textbackslash frac\{\textbackslash infty\}\{\textbackslash infty\}型洛必达法则}}\label{ux221eux221euxaux578bux6d1bux5fc5ux8fbeux6cd5ux5219}}

\hypertarget{ux6cf0ux52d2ux516cux5f0f}{%
\subsubsection{泰勒公式}\label{ux6cf0ux52d2ux516cux5f0f}}

\hypertarget{ux539fux59cbux6cf0ux52d2ux516cux5f0f}{%
\paragraph{原始泰勒公式}\label{ux539fux59cbux6cf0ux52d2ux516cux5f0f}}

\hypertarget{ux5e26ux4f69ux4e9aux8bfaux4f59ux9879ux7684ux6cf0ux52d2ux516cux5f0f}{%
\paragraph{带佩亚诺余项的泰勒公式}\label{ux5e26ux4f69ux4e9aux8bfaux4f59ux9879ux7684ux6cf0ux52d2ux516cux5f0f}}

\hypertarget{ux9ea6ux514bux52b3ux6797ux516cux5f0f}{%
\paragraph{麦克劳林公式}\label{ux9ea6ux514bux52b3ux6797ux516cux5f0f}}

\hypertarget{ux6cf0ux52d2ux516cux5f0fux7684ux51e0ux4f55ux610fux4e49}{%
\paragraph{泰勒公式的几何意义}\label{ux6cf0ux52d2ux516cux5f0fux7684ux51e0ux4f55ux610fux4e49}}

\hypertarget{ux5e38ux7528ux6cf0ux52d2ux516cux5f0f}{%
\paragraph{常用泰勒公式}\label{ux5e38ux7528ux6cf0ux52d2ux516cux5f0f}}

\hypertarget{ux4e0dux5b9aux79efux5206}{%
\subsection{不定积分}\label{ux4e0dux5b9aux79efux5206}}

\hypertarget{ux4e0dux5b9aux79efux5206ux7684ux6982ux5ff5ux4e0eux6027ux8d28}{%
\subsubsection{不定积分的概念与性质}\label{ux4e0dux5b9aux79efux5206ux7684ux6982ux5ff5ux4e0eux6027ux8d28}}

\hypertarget{ux539fux51fdux6570ux4e0eux4e0dux5b9aux79efux5206ux7684ux5b9aux4e49}{%
\paragraph{原函数与不定积分的定义}\label{ux539fux51fdux6570ux4e0eux4e0dux5b9aux79efux5206ux7684ux5b9aux4e49}}

\hypertarget{ux4e0dux5b9aux79efux5206ux7684ux5b9aux4e49}{%
\paragraph{不定积分的定义}\label{ux4e0dux5b9aux79efux5206ux7684ux5b9aux4e49}}

\hypertarget{ux4e0dux5b9aux79efux5206ux7684ux6027ux8d28}{%
\paragraph{不定积分的性质}\label{ux4e0dux5b9aux79efux5206ux7684ux6027ux8d28}}

\hypertarget{ux5e38ux7528ux79efux5206ux8868}{%
\paragraph{常用积分表}\label{ux5e38ux7528ux79efux5206ux8868}}

\hypertarget{ux6362ux5143ux79efux5206ux6cd5-1}{%
\subsubsection{换元积分法}\label{ux6362ux5143ux79efux5206ux6cd5-1}}

\hypertarget{ux7b2cux4e00ux7c7bux6362ux5143ux6cd5}{%
\paragraph{第一类换元法}\label{ux7b2cux4e00ux7c7bux6362ux5143ux6cd5}}

\hypertarget{ux7b2cux4e8cux7c7bux6362ux5143ux6cd5}{%
\paragraph{第二类换元法}\label{ux7b2cux4e8cux7c7bux6362ux5143ux6cd5}}

\hypertarget{ux5206ux90e8ux79efux5206ux6cd5-1}{%
\subsubsection{分部积分法}\label{ux5206ux90e8ux79efux5206ux6cd5-1}}

\hypertarget{ux6709ux7406ux51fdux6570ux79efux5206ux6cd5}{%
\subsubsection{有理函数积分法}\label{ux6709ux7406ux51fdux6570ux79efux5206ux6cd5}}

\hypertarget{ux6709ux7406ux51fdux6570ux62c6ux5206}{%
\paragraph{有理函数拆分}\label{ux6709ux7406ux51fdux6570ux62c6ux5206}}

\hypertarget{ux53efux5316ux4e3aux6709ux7406ux51fdux6570ux79efux5206ux7684ux7c7bux578bux4e07ux80fdux516cux5f0fux66ffux6362}{%
\paragraph{可化为有理函数积分的类型(万能公式替换)}\label{ux53efux5316ux4e3aux6709ux7406ux51fdux6570ux79efux5206ux7684ux7c7bux578bux4e07ux80fdux516cux5f0fux66ffux6362}}

\hypertarget{ux6839ux5f0fux6362ux5143ux5012ux5e26ux6362}{%
\paragraph{根式换元,倒带换}\label{ux6839ux5f0fux6362ux5143ux5012ux5e26ux6362}}

\hypertarget{ux5b9aux79efux5206}{%
\subsection{定积分}\label{ux5b9aux79efux5206}}

\hypertarget{ux5b9aux79efux5206ux7684ux6982ux5ff5ux4e0eux6027ux8d28}{%
\subsubsection{定积分的概念与性质}\label{ux5b9aux79efux5206ux7684ux6982ux5ff5ux4e0eux6027ux8d28}}

\hypertarget{ux5b9aux79efux5206ux7684ux5b9aux4e49}{%
\paragraph{定积分的定义}\label{ux5b9aux79efux5206ux7684ux5b9aux4e49}}

\hypertarget{ux53efux79efux6761ux4ef6ux7684ux5145ux5206ux6761ux4ef6}{%
\paragraph{可积条件的充分条件}\label{ux53efux79efux6761ux4ef6ux7684ux5145ux5206ux6761ux4ef6}}

\hypertarget{ux51e0ux4f55ux610fux4e49}{%
\paragraph{几何意义}\label{ux51e0ux4f55ux610fux4e49}}

\hypertarget{ux5b9aux79efux5206ux7684ux6027ux8d28}{%
\paragraph{定积分的性质}\label{ux5b9aux79efux5206ux7684ux6027ux8d28}}

\hypertarget{ux79efux5206ux4e2dux503cux5b9aux7406-1}{%
\paragraph{积分中值定理}\label{ux79efux5206ux4e2dux503cux5b9aux7406-1}}

\hypertarget{ux5faeux79efux5206ux5b66ux7684ux57faux672cux5b9aux7406}{%
\subsubsection{微积分学的基本定理}\label{ux5faeux79efux5206ux5b66ux7684ux57faux672cux5b9aux7406}}

\hypertarget{ux53d8ux9650ux79efux5206ux5b9aux4e49}{%
\paragraph{变限积分定义}\label{ux53d8ux9650ux79efux5206ux5b9aux4e49}}

\hypertarget{ux53d8ux9650ux79efux5206ux6027ux8d28}{%
\paragraph{变限积分性质}\label{ux53d8ux9650ux79efux5206ux6027ux8d28}}

\hypertarget{ux725bux987f-ux83b1ux5e03ux5c3cux5179ux516cux5f0fux5faeux79efux5206ux57faux672cux5b9aux7406}{%
\paragraph{牛顿-莱布尼兹公式(微积分基本定理)}\label{ux725bux987f-ux83b1ux5e03ux5c3cux5179ux516cux5f0fux5faeux79efux5206ux57faux672cux5b9aux7406}}

\hypertarget{ux5b9aux79efux5206ux7684ux8ba1ux7b97ux65b9ux6cd5}{%
\subsubsection{定积分的计算方法}\label{ux5b9aux79efux5206ux7684ux8ba1ux7b97ux65b9ux6cd5}}

\hypertarget{ux6362ux5143ux79efux5206ux6cd5-2}{%
\paragraph{换元积分法}\label{ux6362ux5143ux79efux5206ux6cd5-2}}

\hypertarget{ux5206ux90e8ux79efux5206ux6cd5-2}{%
\paragraph{分部积分法}\label{ux5206ux90e8ux79efux5206ux6cd5-2}}

\hypertarget{ux5faeux5206ux4e2dux503cux5b9aux7406ux8bc1ux660e}{%
\subparagraph{微分中值定理证明}\label{ux5faeux5206ux4e2dux503cux5b9aux7406ux8bc1ux660e}}

\hypertarget{ux534eux91ccux58ebux516cux5f0f}{%
\subparagraph{华里士公式}\label{ux534eux91ccux58ebux516cux5f0f}}

\hypertarget{ux5b9aux79efux5206ux5b9aux4e49ux8ba1ux7b97ux6781ux9650}{%
\paragraph{定积分定义计算极限}\label{ux5b9aux79efux5206ux5b9aux4e49ux8ba1ux7b97ux6781ux9650}}

\hypertarget{ux53cdux5e38ux79efux5206}{%
\subsubsection{反常积分}\label{ux53cdux5e38ux79efux5206}}

\hypertarget{ux53cdux5e38ux79efux5206ux5b9aux4e49}{%
\paragraph{反常积分定义}\label{ux53cdux5e38ux79efux5206ux5b9aux4e49}}

\hypertarget{ux53cdux5e38ux79efux5206ux7684ux655bux6563ux6027}{%
\paragraph{反常积分的敛散性}\label{ux53cdux5e38ux79efux5206ux7684ux655bux6563ux6027}}

\hypertarget{ux65e0ux754cux7684ux53cdux5e38ux79efux5206ux7455ux70b9}{%
\paragraph{无界的反常积分(瑕点)}\label{ux65e0ux754cux7684ux53cdux5e38ux79efux5206ux7455ux70b9}}

\hypertarget{ux5b9aux79efux5206ux7684ux8fd0ux7528}{%
\subsubsection{定积分的运用}\label{ux5b9aux79efux5206ux7684ux8fd0ux7528}}

\hypertarget{ux5e73ux9762ux56feux5f62ux7684ux9762ux79ef}{%
\paragraph{平面图形的面积}\label{ux5e73ux9762ux56feux5f62ux7684ux9762ux79ef}}

\hypertarget{ux65cbux8f6cux4f53ux4f53ux79ef}{%
\paragraph{旋转体体积}\label{ux65cbux8f6cux4f53ux4f53ux79ef}}

\hypertarget{ux5e38ux5faeux5206ux65b9ux7a0b}{%
\subsection{常微分方程}\label{ux5e38ux5faeux5206ux65b9ux7a0b}}

\hypertarget{ux5faeux5206ux65b9ux7a0bux7684ux57faux672cux6982ux5ff5}{%
\subsubsection{微分方程的基本概念}\label{ux5faeux5206ux65b9ux7a0bux7684ux57faux672cux6982ux5ff5}}

\hypertarget{ux5404ux79cdux89e3ux7684ux5b9aux4e49}{%
\paragraph{各种解的定义}\label{ux5404ux79cdux89e3ux7684ux5b9aux4e49}}

\hypertarget{ux4e00ux9636ux5faeux5206ux65b9ux7a0b}{%
\subsubsection{一阶微分方程}\label{ux4e00ux9636ux5faeux5206ux65b9ux7a0b}}

\hypertarget{ux53efux5206ux79bbux53d8ux91cfux65b9ux7a0bux7684ux5f62ux5f0fux548cux89e3ux6cd5}{%
\paragraph{可分离变量方程的形式和解法}\label{ux53efux5206ux79bbux53d8ux91cfux65b9ux7a0bux7684ux5f62ux5f0fux548cux89e3ux6cd5}}

\hypertarget{ux9f50ux6b21ux5faeux5206ux65b9ux7a0bux7684ux5f62ux5f0fux548cux89e3ux6cd5}{%
\paragraph{齐次微分方程的形式和解法}\label{ux9f50ux6b21ux5faeux5206ux65b9ux7a0bux7684ux5f62ux5f0fux548cux89e3ux6cd5}}

\hypertarget{ux4f2fux52aaux5229ux65b9ux7a0bux7684ux5f62ux5f0fux548cux89e3ux6cd5}{%
\paragraph{伯努利方程的形式和解法}\label{ux4f2fux52aaux5229ux65b9ux7a0bux7684ux5f62ux5f0fux548cux89e3ux6cd5}}

\hypertarget{ux53efux964dux89e3ux7684ux5faeux5206ux65b9ux7a0b}{%
\subsubsection{可降解的微分方程}\label{ux53efux964dux89e3ux7684ux5faeux5206ux65b9ux7a0b}}

\hypertarget{ux76f4ux63a5ux79efux5206ux7c7bux578bux7684ux5faeux5206ux65b9ux7a0bux7684ux5f62ux5f0fux548cux89e3ux6cd5}{%
\paragraph{直接积分类型的微分方程的形式和解法}\label{ux76f4ux63a5ux79efux5206ux7c7bux578bux7684ux5faeux5206ux65b9ux7a0bux7684ux5f62ux5f0fux548cux89e3ux6cd5}}

\hypertarget{ux4e0dux663eux542byux7684ux5faeux5206ux65b9ux7a0bux7684ux5f62ux5f0fux548cux89e3ux6cd5}{%
\paragraph{不显含y的微分方程的形式和解法}\label{ux4e0dux663eux542byux7684ux5faeux5206ux65b9ux7a0bux7684ux5f62ux5f0fux548cux89e3ux6cd5}}

\hypertarget{ux4e0dux663eux793axux7684ux5faeux5206ux65b9ux7a0bux7684ux5f62ux5f0fux548cux89e3ux6cd5}{%
\paragraph{不显示x的微分方程的形式和解法}\label{ux4e0dux663eux793axux7684ux5faeux5206ux65b9ux7a0bux7684ux5f62ux5f0fux548cux89e3ux6cd5}}

\hypertarget{ux9ad8ux9636ux7ebfux6027ux5faeux5206ux65b9ux7a0b}{%
\subsubsection{高阶线性微分方程}\label{ux9ad8ux9636ux7ebfux6027ux5faeux5206ux65b9ux7a0b}}

\hypertarget{ux4e8cux9636ux5e38ux7cfbux6570ux9f50ux6b21ux5faeux5206ux65b9ux7a0b}{%
\paragraph{二阶常系数齐次微分方程}\label{ux4e8cux9636ux5e38ux7cfbux6570ux9f50ux6b21ux5faeux5206ux65b9ux7a0b}}

\hypertarget{ux4e8cux9636ux5e38ux7cfbux6570ux975eux9f50ux6b21ux5faeux98ceux65b9ux7a0b}{%
\paragraph{二阶常系数非齐次微风方程}\label{ux4e8cux9636ux5e38ux7cfbux6570ux975eux9f50ux6b21ux5faeux98ceux65b9ux7a0b}}

\hypertarget{ux89e3ux7684ux6027ux8d28}{%
\paragraph{解的性质}\label{ux89e3ux7684ux6027ux8d28}}

\hypertarget{ux53e0ux52a0ux539fux7406}{%
\paragraph{叠加原理}\label{ux53e0ux52a0ux539fux7406}}

\hypertarget{ux4e8cux9636ux5e38ux7cfbux6570ux7ebfux6027ux975eux9f50ux6b21ux65b9ux7a0bux7279ux89e3ux7684ux5f62ux5f0f}{%
\paragraph{二阶常系数线性非齐次方程特解的形式}\label{ux4e8cux9636ux5e38ux7cfbux6570ux7ebfux6027ux975eux9f50ux6b21ux65b9ux7a0bux7279ux89e3ux7684ux5f62ux5f0f}}

\begin{longtable}[]{@{}ll@{}}
\toprule
\(f(x)\)的形式 & 特解\(y^{*}\)的形式 \\
\midrule
\endhead
& \\
& \\
& \\
& \\
& \\
& \\
& \\
\bottomrule
\end{longtable}

\hypertarget{ux5faeux5206ux7b97ux5b50ux6cd5}{%
\paragraph{微分算子法}\label{ux5faeux5206ux7b97ux5b50ux6cd5}}

\hypertarget{ux9ad8ux9636ux5e38ux7cfbux6570ux9f50ux6b21ux65b9ux7a0bux7684ux5f62ux5f0fux548cux89e3ux6cd5}{%
\subsubsection{高阶常系数齐次方程的形式和解法}\label{ux9ad8ux9636ux5e38ux7cfbux6570ux9f50ux6b21ux65b9ux7a0bux7684ux5f62ux5f0fux548cux89e3ux6cd5}}

\hypertarget{ux7a7aux95f4ux89e3ux6790ux51e0ux4f55}{%
\subsection{空间解析几何}\label{ux7a7aux95f4ux89e3ux6790ux51e0ux4f55}}

\hypertarget{ux5411ux91cfux53caux5176ux8fd0ux7b97}{%
\subsubsection{向量及其运算}\label{ux5411ux91cfux53caux5176ux8fd0ux7b97}}

\hypertarget{ux5e73ux9762ux4e0eux76f4ux7ebf}{%
\subsubsection{平面与直线}\label{ux5e73ux9762ux4e0eux76f4ux7ebf}}

\hypertarget{ux7a7aux95f4ux66f2ux9762ux66f2ux7ebfux53caux5176ux65b9ux7a0b}{%
\subsubsection{空间曲面、曲线及其方程}\label{ux7a7aux95f4ux66f2ux9762ux66f2ux7ebfux53caux5176ux65b9ux7a0b}}

\hypertarget{ux591aux5143ux51fdux6570ux5faeux5206ux5b66}{%
\subsection{多元函数微分学}\label{ux591aux5143ux51fdux6570ux5faeux5206ux5b66}}

\hypertarget{ux591aux5143ux51fdux6570ux7684ux6781ux9650ux4e0eux8fdeux7eed}{%
\subsubsection{多元函数的极限与连续}\label{ux591aux5143ux51fdux6570ux7684ux6781ux9650ux4e0eux8fdeux7eed}}

\hypertarget{ux504fux5bfcux6570}{%
\subsubsection{偏导数}\label{ux504fux5bfcux6570}}

\hypertarget{ux5168ux5faeux5206}{%
\subsubsection{全微分}\label{ux5168ux5faeux5206}}

\hypertarget{ux591aux5143ux51fdux6570ux6781ux503c}{%
\subsubsection{多元函数极值}\label{ux591aux5143ux51fdux6570ux6781ux503c}}

\hypertarget{ux591aux5143ux51fdux6570ux5faeux5206ux5b66ux7684ux51e0ux4f55ux5e94ux7528}{%
\subsubsection{多元函数微分学的几何应用}\label{ux591aux5143ux51fdux6570ux5faeux5206ux5b66ux7684ux51e0ux4f55ux5e94ux7528}}

\hypertarget{ux7a7aux95f4ux66f2ux7ebfux548cux6cd5ux5e73ux9762}{%
\paragraph{空间曲线和法平面}\label{ux7a7aux95f4ux66f2ux7ebfux548cux6cd5ux5e73ux9762}}

\hypertarget{ux53c2ux6570ux5f0f}{%
\subparagraph{参数式}\label{ux53c2ux6570ux5f0f}}

设空间曲线\(\Gamma\)是由参数方程\(\left\{
\begin{aligned}
x & = x(t) \\
y & = y(t) \\
z & = z(t) \\
\end{aligned}
\right.\)确定,该式中的三个函数均可导,且导数不同时为零,设\(M_0(x_0,y_0,z_0)\)为曲线\(\Gamma \)上一点,对应\(t = t_0\),则曲线在\(M_0\)处,

\begin{enumerate}
\def\labelenumi{\arabic{enumi}.}
\item
  切向量为:\(\vec{T}= (x^{'}(t_0),y^{'}(t_0),z^{'}(t_0));\)
\item
  切线方程为:\(\frac{x-x_0}{x^{'}(t_0)},\frac{y-y_0}{y^{'}(t_0)},\frac{z-z_0}{z^{'}(t_0)}\);
\item
  法平面方程为:
  \(x^{'}(t_0)(x-x_0)+y^{'}(t_0)(y-y_0)+z^{'}(t_0)(z-z_0)=0\);
\end{enumerate}

\hypertarget{ux76f8ux4ea4ux5f0f}{%
\subparagraph{相交式}\label{ux76f8ux4ea4ux5f0f}}

空间曲线由相交式方程\(\left\{ \begin{aligned} F(x,y,z) & = 0 \\ G(x,y,z) & = 0 \end{aligned} \right.\),所确定,设\(M_0(x_0,y_0,z_0)\)为曲线\(\Gamma\)上一点,对应\(t=t_0\),则曲线在\(M_0\)处,

\begin{enumerate}
\def\labelenumi{\arabic{enumi}.}
\item
  切向量为\(\vec{T} = \begin{pmatrix} \begin{vmatrix} F^{'}_y & F^{'}_z \\ G^{'}_y & G^{'}_z \end{vmatrix},\begin{vmatrix} F^{'}_z & F^{'}_x \\ G^{'}_z & G^{'}_x\end{vmatrix},\begin{vmatrix} F^{'}_x & F^{'}_y \\ G^{'}_y & G^{'}_z \end{vmatrix} \end{pmatrix}\);
\item
  切向方程为\( \frac{x-x_0}{\begin{vmatrix} F^{'}_y & F^{'}_z \\ G^{'}_y & G^{'}_z \end{vmatrix}}=\frac{y-y_0}{\begin{vmatrix} F^{'}_z & F^{'}_x \\ G^{'}_z & G^{'}_x\end{vmatrix}}=\frac{z-z_0}{\begin{vmatrix} F^{'}_x & F^{'}_y \\ G^{'}_y & G^{'}_z \end{vmatrix}}\);
\item
  法平面方程为:\( \begin{vmatrix} F^{'}_y & F^{'}_z \\ G^{'}_y & G^{'}_z \end{vmatrix}(x-x_0)+\begin{vmatrix} F^{'}_z & F^{'}_x \\ G^{'}_z & G^{'}_x\end{vmatrix}(y-y_0)+\begin{vmatrix} F^{'}_x & F^{'}_y \\ G^{'}_y & G^{'}_z \end{vmatrix}(z-z_0)=0\);
\end{enumerate}

\hypertarget{ux66f2ux9762ux7684ux5207ux5e73ux9762ux4e0eux6cd5ux7ebf}{%
\paragraph{曲面的切平面与法线}\label{ux66f2ux9762ux7684ux5207ux5e73ux9762ux4e0eux6cd5ux7ebf}}

曲面\(\Sigma\)是由方程\(F(x,y,z)=0 \)所确定,\(F\)有连续的偏导数,\(M_0(x_0,y_0,z_0)\)是曲面上一点,设曲面方程为\(F(x,y,z)=0\)在曲面上任取一条通过\(M_0(x_0,y_0,z_0)\)的曲线\(\Gamma : \left\{ \begin{aligned} x & = x(t) \\ y & = y(t) \\ z & = z(t) \\ \end{aligned} \right.\),曲线在\(M\)处的切向量\(\vec{T}=(x^{'} (x_0),y^{'} (y_0),z^{'} (z_0))\),令\(\vec{n} = (\vec{F^{'}_x(x_0,y_0,z_0)},\vec{F^{'}_y(x_0,y_0,z_0)},\vec{F^{'}_z(x_0,y_0,z_0)})\),则\(\vec{n} \perp \vec{T}\),由于曲线是曲面上通过\(M\)的任意一条曲线,它们在\(M\)的切线都与同一向量\(\vec{n}\)垂直,故曲面上通过\(M\)的一切曲线在点\(M\)的切线都在同一平面上,这个平面称为在点\(M\)的切平面。则在点\(M_0\)处,

\begin{enumerate}
\def\labelenumi{\arabic{enumi}.}
\item
  法向量
  \(\vec{n} = (\vec{F^{'}_x(x_0,y_0,z_0)},\vec{F^{'}_y(x_0,y_0,z_0)},\vec{F^{'}_z(x_0,y_0,z_0)})\);
\item
  切平面方程为\(\vec{F^{'}_x(x_0,y_0,z_0)}(x-x_0)+\vec{F^{'}_y(x_0,y_0,z_0)}(y-y_0)+\vec{F^{'}_z(x_0,y_0,z_0)}(z-z_0)\);
\item
  该切平面的法线方程为:\(\frac{(x-x_0)}{\vec{F^{'}_x(x_0,y_0,z_0)}}+\frac{(y-y_0)}{\vec{F^{'}_y(x_0,y_0,z_0)}}+\frac{(z-z_0)}{\vec{F^{'}_z(x_0,y_0,z_0)}}\);
\end{enumerate}

\hypertarget{ux7a7aux95f4ux66f2ux9762ux65b9ux7a0bux5f62ux4e3auxazfxy}{%
\subparagraph{\texorpdfstring{空间曲面方程形为\(z=f(x,y)\)}{空间曲面方程形为z=f(x,y)}}\label{ux7a7aux95f4ux66f2ux9762ux65b9ux7a0bux5f62ux4e3auxazfxy}}

令\(F(x,y,z)=f(x,y)-z\),\(f\)有连续的偏导数,则曲面在\(M_0\)处,

\begin{enumerate}
\def\labelenumi{\arabic{enumi}.}
\item
  法向量为\(\vec{n}=(-f^{'}_x (x_0,y_0),-f^{'}_y (x_0,y_0),1)\);
\item
  法线方程为\(\frac{x-x_0}{-f^{'}_x (x_0,y_0)}=\frac{y-y_0}{-f^{'}_y (x_0,y_0)}=\frac{z-z_0}{1}\);
\item
  切平面方程为\(-f^{'}_x (x_0,y_0)(x-x_0)-f^{'}_y (x_0,y_0)(y-y_0)+(z-z_0)=0\);
\end{enumerate}

\hypertarget{ux65b9ux5411ux5bfcux6570ux4e0eux68afux5ea6}{%
\subsubsection{方向导数与梯度}\label{ux65b9ux5411ux5bfcux6570ux4e0eux68afux5ea6}}

\hypertarget{ux65b9ux5411ux5bfcux6570ux7684ux5b9aux4e49}{%
\paragraph{方向导数的定义}\label{ux65b9ux5411ux5bfcux6570ux7684ux5b9aux4e49}}

\textbf{问题引入:}

\(z=f(x,y)\)沿着横轴\(x\)的导数为\(\frac{\partial{f}}{\partial{x}}\),沿着\(y\)的导数为\(\frac{\partial{f}}{\partial{y}}\),那如果不是沿着这两个方向,而是沿着任意方向\(\vec{e_t} =(\cos\alpha,\cos\beta)\)求导,应该如何计算呢?

\textbf{方向导数定义}

设函数\(z = f(x,y)\)在点\(M_0(x_0,y_0)\)的某一领域内有定义,从点\(M_0\)引入一条射线\(l\),设于\(l\)方向相同的单位向量为\(\vec{e}=(\cos \alpha,\cos\beta)\),则射线\(l\)的参数方程为\(\left \{ \begin{aligned} x = x_0 + t\cos\alpha \\y=y_0+t\cos\beta \end{aligned} (t\ge 0) \right.\).

设\(M_1(x_0+t\cos\alpha,y_0+t\cos\beta)\)为\(l\)上另一点,当\(t \rarr 0\)时,函数增量\(f(x_0+t\cos\alpha,y_0+t\cos\beta)-f(x_0,y_0)\)与\(M1\)到\(M_0\)的距离\(\vert M_0M_1 \vert = t\)的比值\(\frac{f(x_0+t\cos\alpha,y_0+t\cos\beta)-f(x_0,y_0)}{t}\)的极限存在,那么称该极限为函数\(f(x,y)\)在点\(M_0\)沿方向\(l\)的方向导数,记作\(\frac{\partial{f}}{\partial{l}}\vert_{(x_0,y_0)}\),即\(\frac{\partial{f}}{\partial{l}}\vert_{(x_0,y_0)} = \lim\limits_{t \rarr 0^+}\frac{f(x_0+t\cos\alpha,y_0+t\cos\beta)-f(x_0,y_0)}{t}\)

\hypertarget{ux68afux5ea6ux7684ux5b9aux4e49}{%
\paragraph{梯度的定义}\label{ux68afux5ea6ux7684ux5b9aux4e49}}

设函数\(z=f(x,y)\)在平面区域\(D\)内具有一阶连续偏导数,则对于每一点\(M(x,y) \in D\),都可定出一个向量\(\frac{\partial{f}}{\partial{x}} \vec{i}+ \frac{\partial{f}}{\partial{y}} \vec{j}\),这向量称为函数\(z=f(x,y)\)在点\(M(x,y)\)的梯度,记为\(\nabla f(x,y)= \frac{\partial{f}}{\partial{x}} \vec{i}+ \frac{\partial{f}}{\partial{y}} \vec{j}\)\footnote{这里的\(\vec{i}和\vec{j}使得梯度的方向得以改变\)},则\(\frac{\partial{f}}{\partial{l}}=\frac{\partial{f}}{\partial{x}} \cos\alpha+ \frac{\partial{f}}{\partial{y}} \cos\beta = (\frac{\partial{f}}{\partial{x}},\frac{\partial{f}}{\partial{y}})\cdot (\cos\alpha,cos\beta)=\nabla f(x,y)\cdot\vec{e_t}=\vert \nabla f(x,y) \vert \cos \theta\),其中\(\theta\)为\(\nabla f(x,y)\)与\(\vec{e_t}\)的角度。

\hypertarget{ux91cdux79efux5206}{%
\subsection{重积分}\label{ux91cdux79efux5206}}

\hypertarget{ux4e8cux91cdux79efux5206ux7684ux6982ux5ff5ux4e0eux6027ux8d28}{%
\subsubsection{二重积分的概念与性质}\label{ux4e8cux91cdux79efux5206ux7684ux6982ux5ff5ux4e0eux6027ux8d28}}

\hypertarget{ux4e8cux91cdux79efux5206ux7684ux5b9aux4e49}{%
\paragraph{二重积分的定义}\label{ux4e8cux91cdux79efux5206ux7684ux5b9aux4e49}}

设\(f(x,y)\)是有界闭区域\(D\)上的有界函数,将闭区域\(D\)任意划分成小闭区域\(\Delta\sigma_1,\Delta\sigma_2,\cdots,\Delta\sigma_n\),其中\(\Delta \sigma_i\)表示第\(i\)个小闭区域,也表示它的面积,在每一个\(\Delta \sigma_i\)上任取一点\((\xi_i,\eta_i)\Delta \sigma_i\)上任取一点(\(\xi_i,\eta_i\))
,作乘积\(f(\xi_i,\eta_i)\Delta\sigma_i\),并作和\(\sum\limits_{i=1}^{n}f(\xi_i,\eta_i)\Delta \sigma_i\).如果当各个小闭区间的直径中的最大值\(\lambda\rarr 0\)时,\(\sum\limits_{i=1}^{n}f(\xi_i,\eta_i)\Delta \sigma_i\)的极限总存在,且与闭区域\(D\)的分法及点\(\xi_i,\eta_i\)的取法无关,那么称此极限为函数\(f(x,y)\)的闭区域\(D\)上的二重积分,记作\(\iint\limits_D f(x,y) d\sigma\),即\(\iint\limits_D f(x,y) d\sigma=\lim\limits_{i=1} \sum\limits_{i=1}^n f(\xi_i,\eta_i)\Delta\sigma_i\).

\(\iint\limits_D f(x,y) d\sigma \)中,\(f(x,y)\)叫做被积函数,\(d\sigma\)叫做面积元素,\(x\)与\(y\)叫做积分变量,\(D\)叫做积分区域。

\hypertarget{ux4e8cux91cdux79efux5206ux7684ux6027ux8d28}{%
\paragraph{二重积分的性质}\label{ux4e8cux91cdux79efux5206ux7684ux6027ux8d28}}

\hypertarget{ux7ebfux6027ux6027ux8d28}{%
\subparagraph{线性性质:}\label{ux7ebfux6027ux6027ux8d28}}

\[\iint\limits_D kf(x,y) d\sigma = k \iint\limits_Df(x,y)\]

\[\iint\limits_D[f(x,y)\pm g(x,y)]d\sigma = \iint\limits_Df(x,y)d\sigma \pm \iint\limits_Dg(x,y)d\sigma\]

\hypertarget{ux5bf9ux533aux57dfux7684ux53efux52a0ux6027}{%
\subparagraph{对区域的可加性}\label{ux5bf9ux533aux57dfux7684ux53efux52a0ux6027}}

\[\iint\limits_Df(x,y)d\sigma = \iint\limits_{D_1}f(x,y)d\sigma + \iint\limits_{D_2}g(x,y)d\sigma \\(D=D_1+D_2)\]

\hypertarget{ux82e5uxasduxaux4e3auxaduxaux7684ux9762ux79efuxaux222cddux3c3sd}{%
\subparagraph{\texorpdfstring{若\(S_D\)为\(D\)的面积,\(\iint\limits_Dd\sigma=S_D\)}{若S\_D为D的面积,\textbackslash iint\textbackslash limits\_Dd\textbackslash sigma=S\_D}}\label{ux82e5uxasduxaux4e3auxaduxaux7684ux9762ux79efuxaux222cddux3c3sd}}

\hypertarget{ux6bd4ux8f83ux6027ux8d28}{%
\subparagraph{比较性质}\label{ux6bd4ux8f83ux6027ux8d28}}

\begin{enumerate}
\def\labelenumi{\arabic{enumi}.}
\item
  若函数\(f(x,y)\)与\(g(x,y)\)在区域\(D\)上可积,且具有\(f(x,y)\le g(x,y)\)在区域\(D\)上成立,则有\(\iint\limits_D f(x,y)\le \iint\limits_Dg(x,y)\).
\item
  若函数\(f(x,y)\)与\(g(x,y)\)在区域\(D\)上可积,且具有\(f(x,y)\le g(x,y)\),但\(f(x,y)\)不恒等于\(g(x,y)\),则有\(\iint\limits_D f(x,y) < \iint\limits_Dg(x,y)\).

  \begin{itemize}
  \item
    特殊地\(\vert \iint\limits_D f(x,y)d\sigma \vert \le \iint\limits\vert f(x,y) \vert d\sigma(-\iint\limits\vert f(x,y) \vert d\sigma\le \iint\limits_D f(x,y)d\sigma\le \iint\limits\vert f(x,y) \vert d\sigma)\).
  \end{itemize}
\end{enumerate}

\hypertarget{ux79efux5206ux4f30ux503cux5b9aux7406}{%
\subparagraph{积分估值定理}\label{ux79efux5206ux4f30ux503cux5b9aux7406}}

设\(M,m\)分别是\(f(x,y)\)在闭区域\(D\)上的最大值和最小值,\(S_D\)为\(D\)的面积,则\(mS_D\le\iint\limits_D f(x,y) d\sigma\le MS_D\).

\hypertarget{ux79efux5206ux4e2dux503cux5b9aux7406-2}{%
\subparagraph{⭐️积分中值定理}\label{ux79efux5206ux4e2dux503cux5b9aux7406-2}}

设函数\(f(x,y)\)在闭区间\(D\)上连续,\(S_D\)为\(D\)的面积,则在\(D\)上至少存在一点\((\xi,\eta)\)使得\(\iint\limits_Df(x,y)d\sigma=f(\xi,\eta)S_D\).

\begin{center}\rule{0.5\linewidth}{0.5pt}\end{center}

\hypertarget{ux4e8cux91cdux79efux5206ux7684ux8ba1ux7b97}{%
\subsubsection{二重积分的计算}\label{ux4e8cux91cdux79efux5206ux7684ux8ba1ux7b97}}

\hypertarget{ux76f4ux89d2ux5750ux6807ux8ba1ux7b97ux4e8cux91cdux79efux5206}{%
\paragraph{直角坐标计算二重积分}\label{ux76f4ux89d2ux5750ux6807ux8ba1ux7b97ux4e8cux91cdux79efux5206}}

\hypertarget{ux4e8cux91cdux79efux5206ux5316ux4e3aux7d2fux6b21ux79efux5206}{%
\subparagraph{二重积分化为累次积分}\label{ux4e8cux91cdux79efux5206ux5316ux4e3aux7d2fux6b21ux79efux5206}}

\(D\)为平面上有界闭区域,\(f(x,y)\)在\(D\)上连续,\(x=x_0\)处截面面积\(A(x_0)=\int_{y_1(x_0)}^{y_2(x_0)}f(x,y)dy\),任一点处截面面积\(A(x)=\int_{y_1(x)}^{y_2(x)}f(x,y)dy\),所以

\[V = \int_a^bA(x)dx =\int_a^b[\int_{y_1(x)}^{y_2(x)}dy]dx\]

\textbf{X型区域}

\(X\)型区域的特点,穿过区域且平行\(y\)轴的直线与区域边界相交不多于两个交点。

\(X\)型区域\(D\)可以用不等式表示为\(\{(x,y)\vert a \le x \le  b,y_1(x)\le y\le y_2(x)\}\).

其中函数\(y_1(x),y_2(y)\)在区域\([a,b]\)上连续,则\(\iint\limits_Df(x,y)d\sigma=\int_a^bdx\int_{y_1(x)}^{y_2(x)}f(x,y)dy\).

\textbf{Y型区域}

\(Y\)型区域的特点,穿过区域且平行\(x\)轴的直线与区域边界相交不多于两个交点。

\(Y\)型区域\(D\)可以用不等式表示为\(\{(x,y)\vert c \le y \le  d,x_1(x)\le x\le x_2(x)\}\).

其中函数\(y_1(x),y_2(y)\)在区域\([a,b]\)上连续,则\(\iint\limits_Df(x,y)d\sigma=\int_c^ddy\int_{x_1(y)}^{x_2(y)}f(x,y)dx\).

\hypertarget{ux6781ux5750ux6807ux8ba1ux7b97ux4e8cux91cdux79efux5206-1}{%
\paragraph{极坐标计算二重积分}\label{ux6781ux5750ux6807ux8ba1ux7b97ux4e8cux91cdux79efux5206-1}}

\hypertarget{ux6781ux5750ux6807ux7684ux6982ux5ff5}{%
\subparagraph{极坐标的概念}\label{ux6781ux5750ux6807ux7684ux6982ux5ff5}}

极坐标也可以表示平面内的任何一个点\((x,y)\),但只有一个坐标轴:极轴.

极轴是从平面中选取一个点作为极点\((0,0)\)(类似于直角坐标系中的原点)后,从极点引出的一条射线,一般选择的方向和直角坐标系中的x轴一样,右方为正方向.

极径\(\rho\)表示任一点到极点的距离、极角\(\theta\)正为逆时针,负为顺时针。

\hypertarget{ux6781ux5750ux6807ux4e0eux76f4ux89d2ux5750ux6807ux7684ux8f6cux6362}{%
\subparagraph{极坐标与直角坐标的转换}\label{ux6781ux5750ux6807ux4e0eux76f4ux89d2ux5750ux6807ux7684ux8f6cux6362}}

\[\left \{ 
\begin{aligned} x & = \rho \cos\theta \\y & =\rho \sin\theta \end{aligned} \right.,\sqrt{x^2+y^2}=\rho,\tan\theta = \frac{y}{x}.\]

\hypertarget{ux6781ux5750ux6807ux8ba1ux7b97ux4e8cux91cdux79efux5206-2}{%
\subparagraph{极坐标计算二重积分}\label{ux6781ux5750ux6807ux8ba1ux7b97ux4e8cux91cdux79efux5206-2}}

积分上下限范围:

\begin{enumerate}
\def\labelenumi{\arabic{enumi}.}
\item
  \(\theta \)找最大值和最小值.
\item
  在\(\theta\)的范围内,从极点发出射线,与平面区域\(D\)的边界最多只能有两个交点,先穿入的曲线\(\rho_1(\theta)\)为\(\rho\)的下限,后穿出的曲线\(\rho_2(\theta)\)为\(\rho\)的上限.
\end{enumerate}

极坐标转换公式

\[\iint\limits_D(x,y)dxdy = \iint\limits f(\rho\cos\theta,\rho\sin\rho)\rho d\rho d\theta\]

\textbf{注:}

多出一个\(\rho\)是从雅可比行列式中所得,后补充。

\hypertarget{ux4e8cux91cdux79efux5206ux7684ux5bf9ux79f0ux6027}{%
\subsubsection{二重积分的对称性}\label{ux4e8cux91cdux79efux5206ux7684ux5bf9ux79f0ux6027}}

\hypertarget{ux5947ux500dux5076ux96f6}{%
\paragraph{奇倍偶零}\label{ux5947ux500dux5076ux96f6}}

设\(f(x,y)\)在有界闭区域\(D\)上连续。

\begin{enumerate}
\def\labelenumi{\arabic{enumi}.}
\item
  积分区域\(D\)关于\(y\)轴对称(\(y\)对称\(x\)奇偶),则:

  \[\iint\limits_D f(x,y)dxdy = \left \{ \begin{aligned}& 2\iint\limits_{D_1}f(x,y)dxdy,f(-x,y)=f(x,y) \\ &0,f(-x,y)=-f(x,y)\end{aligned} \right.(其中D_1是D的右半边)\]
\item
  积分区域\(D\)关于\(x\)轴对称(\(x\)对称\(y\)奇偶),则:

  \[\iint\limits_D f(x,y)dxdy = \left \{ \begin{aligned}& 2\iint\limits_{D_1}f(x,y)dxdy,f(x,-y)=f(x,y) \\ &0,f(x,-y)=-f(x,y)\end{aligned} \right.(其中D_1是D的上半边)\]
\item
  积分区域\(D\)关于原点对称(\(x\)对称\(y\)奇偶),则:

  \[\iint\limits_D f(x,y)dxdy = \left \{ \begin{aligned}& 2\iint\limits_{D_1}f(x,y)dxdy,f(-x,-y)=f(x,y) \\ &0,f(-x,-y)=-f(x,y)\end{aligned} \right.(其中D_1是D的上半边)\]
\end{enumerate}

\textbf{总结:}

\(x\)对称\(y\)奇偶,\(y\)对称\(x\)奇偶,原点对称\((x,y)\)奇偶。

\hypertarget{ux8f6eux6362ux5bf9ux79f0ux6027}{%
\paragraph{轮换对称性}\label{ux8f6eux6362ux5bf9ux79f0ux6027}}

\begin{enumerate}
\def\labelenumi{\arabic{enumi}.}
\item
  \(D_{xy}\)与\(D_{yx}\)的关系为:一个区域的限定方程中的\(x,y\)互相替换变成了限定另一个区域的方程,但是面积不发生变化,则:

  \[\iint\limits_{D_{xy}} f(x,y) dxdy =\iint\limits_{D_{xy}} f(x,y) dxdy\]
\item
  若积分区域\(D\)关于\(y=x\)对称,则:

  \[\iint\limits_D f(x,y) dxdy = \iint\limits_D f(y,x) dxdy =\frac{1}{2} \iint\limits_D [f(x,y)+f(y,x)] dxdy\]
\item
  若积分区域\(D\)关于\(y=x\)对称,且被积函数\(f(x,y)\)满足\(f(x,y)=f(y,x)\),则\(f(x,y)\)关于\(x\)和\(y\)轮换对称,那么:

  \[\iint\limits_D f(x,y) dxdy =2\iint\limits_{D_1} f(x,y) dxdy \qquad (D_1 = \frac{1}{2} D)\]
\end{enumerate}

\hypertarget{ux4e09ux91cdux79efux5206ux7684ux5b9aux4e49}{%
\subsubsection{三重积分的定义}\label{ux4e09ux91cdux79efux5206ux7684ux5b9aux4e49}}

\(f(x,y,z)\)是空间比区域\(\Omega\)上的有界函数,将\(\Omega\)任意分成\(n\)个闭区域,\(\Delta v_1,\Delta v_2,\cdots,\Delta v_n,\cdots\),\(\Delta v_i\)表示\(i\)个闭区域,也表示它的体积。在每个\(\Delta v_i\)上任取一点\((\xi_i,\eta_i,\zeta_i)\),作乘积\(f(\xi_i,\eta_i,\zeta_i) \ \Delta v_i(i=1,2,\cdots,n)\),并作和\(\sum \limits _{i=1} ^{n}f(\xi_i,\eta_i,\zeta_i)\Delta v_i\),如果让各小闭区域的直径中的最大值\(\lambda\)趋近于零时,这个和的极限总存在(与\(\Delta v_i\)的分法无关及\((\xi_i,\eta_i,\zeta_i)\)的取法无关),则称此极限值为函数\(f(x,y,z)\)在闭区域\(\Omega\)上的三重积分,记作\(\iiint\limits_\Omega f(x,y,z) dv\),即:

\[\iiint\limits_\Omega f(x,y,z) = \lim \limits _{\lambda \to 0} \sum\limits_{i=1}^n f(\xi_i,\eta_i,\zeta_i)\Delta v_i\]

\textbf{注解:}

二重积分计算的是体积,三重积分计算的质量,但是在空间上并不绝对,笔者认为,二重积分只不过是计算一个由两个数据决定的第三个数据在前两个数据不同范围内分布的关系,三重积分计算一个由三个数据决定的第四个数据在前三个数据不同范围内分布的关系,只要符合使用的规律,可以二维可以算面积,三维可以算体积。

\hypertarget{ux4e09ux91cdux79efux5206ux7684ux8ba1ux7b97}{%
\subsubsection{三重积分的计算}\label{ux4e09ux91cdux79efux5206ux7684ux8ba1ux7b97}}

\hypertarget{ux5229ux7528ux76f4ux89d2ux5750ux6807ux8ba1ux7b97ux4e09ux91cdux79efux5206}{%
\paragraph{利用直角坐标计算三重积分}\label{ux5229ux7528ux76f4ux89d2ux5750ux6807ux8ba1ux7b97ux4e09ux91cdux79efux5206}}

\hypertarget{ux5229ux7528ux67f1ux9762ux5750ux6807ux8ba1ux7b97ux4e09ux91cdux79efux5206}{%
\paragraph{利用柱面坐标计算三重积分}\label{ux5229ux7528ux67f1ux9762ux5750ux6807ux8ba1ux7b97ux4e09ux91cdux79efux5206}}

\hypertarget{ux5229ux7528ux7403ux9762ux5750ux6807ux8ba1ux7b97ux4e09ux91cdux79efux5206}{%
\paragraph{利用球面坐标计算三重积分}\label{ux5229ux7528ux7403ux9762ux5750ux6807ux8ba1ux7b97ux4e09ux91cdux79efux5206}}

\hypertarget{ux66f2ux7ebfux79efux5206ux4e0eux66f2ux9762ux79efux5206}{%
\subsection{曲线积分与曲面积分}\label{ux66f2ux7ebfux79efux5206ux4e0eux66f2ux9762ux79efux5206}}

\hypertarget{ux7b2cux4e00ux7c7bux66f2ux7ebfux79efux5206}{%
\subsubsection{第一类曲线积分}\label{ux7b2cux4e00ux7c7bux66f2ux7ebfux79efux5206}}

\hypertarget{ux7b2cux4e8cux7c7bux66f2ux7ebfux79efux5206}{%
\subsubsection{第二类曲线积分}\label{ux7b2cux4e8cux7c7bux66f2ux7ebfux79efux5206}}

\hypertarget{ux683cux6797ux516cux5f0fux53caux5176ux5e94ux7528}{%
\subsubsection{格林公式及其应用}\label{ux683cux6797ux516cux5f0fux53caux5176ux5e94ux7528}}

\hypertarget{ux7b2cux4e00ux7c7bux66f2ux9762ux79efux5206}{%
\subsubsection{第一类曲面积分}\label{ux7b2cux4e00ux7c7bux66f2ux9762ux79efux5206}}

\hypertarget{ux7b2cux4e8cux7c7bux66f2ux9762ux79efux5206}{%
\subsubsection{第二类曲面积分}\label{ux7b2cux4e8cux7c7bux66f2ux9762ux79efux5206}}

\hypertarget{ux9ad8ux65afux516cux5f0f}{%
\subsubsection{高斯公式}\label{ux9ad8ux65afux516cux5f0f}}

\hypertarget{ux65afux6258ux514bux65afux516cux5f0f}{%
\subsubsection{斯托克斯公式}\label{ux65afux6258ux514bux65afux516cux5f0f}}

\hypertarget{ux65e0ux7a77ux7ea7ux6570}{%
\subsection{无穷级数}\label{ux65e0ux7a77ux7ea7ux6570}}

\hypertarget{ux5e38ux6570ux9879ux7ea7ux6570ux7684ux6982ux5ff5ux548cux6027ux8d28}{%
\subsubsection{常数项级数的概念和性质}\label{ux5e38ux6570ux9879ux7ea7ux6570ux7684ux6982ux5ff5ux548cux6027ux8d28}}

\hypertarget{ux5e38ux6570ux9879ux7ea7ux6570ux5ba1ux655bux6cd5}{%
\subsubsection{常数项级数审敛法}\label{ux5e38ux6570ux9879ux7ea7ux6570ux5ba1ux655bux6cd5}}

\hypertarget{ux5e42ux7ea7ux6570}{%
\subsubsection{幂级数}\label{ux5e42ux7ea7ux6570}}

\hypertarget{ux51fdux6570ux5c55ux5f00ux6210ux5e42ux7ea7ux6570}{%
\subsubsection{函数展开成幂级数}\label{ux51fdux6570ux5c55ux5f00ux6210ux5e42ux7ea7ux6570}}

\hypertarget{ux5085ux91ccux53f6ux7ea7ux6570}{%
\subsubsection{傅里叶级数}\label{ux5085ux91ccux53f6ux7ea7ux6570}}

\hypertarget{ux96f6ux788eux77e5ux8bc6}{%
\subsection{零碎知识}\label{ux96f6ux788eux77e5ux8bc6}}

\hypertarget{ux53c2ux6570ux65b9ux7a0b}{%
\subsubsection{参数方程}\label{ux53c2ux6570ux65b9ux7a0b}}

\hypertarget{ux66f2ux7387ux4e0eux5f27ux957f}{%
\subsubsection{曲率与弧长}\label{ux66f2ux7387ux4e0eux5f27ux957f}}

\hypertarget{ux9762ux79efux4e0eux4f53ux79ef}{%
\subsubsection{面积与体积}\label{ux9762ux79efux4e0eux4f53ux79ef}}

\hypertarget{ux7269ux7406ux8fd0ux7528}{%
\subsubsection{物理运用}\label{ux7269ux7406ux8fd0ux7528}}

\end{document}
